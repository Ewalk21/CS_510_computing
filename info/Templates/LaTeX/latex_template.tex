\documentclass[aps,pra,notitlepage,amsmath,amssymb,letterpaper,12pt]{revtex4-1}
\usepackage{amsthm,enumerate}
\usepackage{graphicx}
%  Above uses the Americal Physical Society template for Physical Review A
%  as a reasonable and fully-featured default template

%  Below define helpful commands to set up problem environments easily
\newtheorem{theorem}{Theorem}
\newenvironment{problem}[2][Problem]{\begin{trivlist}
\item[\hskip \labelsep {\bfseries #1}\hskip \labelsep {\bfseries #2.}]}{\end{trivlist}}
\newenvironment{solution}{\begin{proof}[Solution]}{\end{proof}}

% --------------------------------------------------------------
%                   Document Begins Here
% --------------------------------------------------------------

\begin{document}

\title{Midterm Project 4}
\author{Madeline Dumontelle, Jacob Lepp, Alejandra Luna, Evan Walker}
\affiliation{Math 450, Schmid College of Science and Technology, Chapman University}
\date{\today}

\maketitle

\section{Proofs for Theorems 5.12 and 5.13}


\begin{theorem}[Generalized Mean-Value Theorem(5.12)]
Let $f$ and $g$ be two functions, each having a derivative (finite or infinite) at each point
of an open interval $(a,b)$ and each continuous at the endpoints $a$ and $b$. Assume also that there
is no interior point $x$ at which both $f^\prime(x)$ and $g^\prime(x)$ are infinite. Then for some
interior point $c$ we have
\[f^\prime(c)[g(b)-g(a)] = g^\prime(c)[f(b)-f(a)]\]
\end{theorem}

\begin{proof}
proof here
\end{proof}

\begin{theorem}[5.12 in our book]
Let $f$ and $g$ be two functions, each having a derivative (finite or infinite) at each point of $(a,b)$. At the endpoints assume that the limits $f(a+)$,$g(a+)$,$f(b-)$, and $g(b-)$ exist as finite values. Assume further that there is no interior point x at which both $f^\prime(x)$ and $g^\prime(x)$ are infinite. Then for some
interior point $c$ we have
\[f^\prime(c)[g(b-)-g(a+)] = g^\prime(c)[f(b-)-f(a+)]\]
\end{theorem}

\begin{proof}
proof here
\end{proof}

\section{Proofs for Theorems 5.11, 5.14, and 5.16 as Examples}

\begin{theorem}[Mean-Value Theorem(5.11)]
Assume that $f$ has a derivative (finite or infinite) at each point of $(a,b)$, and assume that $f$ is continuous at both endpoints $a$ and $b$. Then $\exists c \in (a,b)$ such that
\[f(b) - f(a) = f^\prime(c)(b-a)\]
\end{theorem}

\begin{proof}
proof here
\end{proof}

\begin{theorem}[5.14 in our book]
Assume that $f$ has a derivative (finite or infinite) at each point of an open interval $(a,b)$, and that $f$ is continuous at both endpoints $a$ and $b$.
\begin{enumerate}[\upshape (a)]
  \item if $f^\prime$ takes only positive values (finite or infinite) in $(a,b)$, then $f$ is strictly increasing on $[a,b]$.
  \item if $f^\prime$ takes only negative values (finite or infinite) in $(a,b)$, then $f$ is strictly decreasing on $[a,b]$.
  \item if $f^\prime$ is zero everywhere in $(a,b)$, then $f$ is constant on $[a,b]$.
\end{enumerate}
\end{theorem}

\begin{proof}
proof here
\end{proof}

\begin{theorem}[Intermediate-value theorem for derivatives(5.16)]
Assume that $f$ is defined on a compact interval $[a,b]$ and that $f$ has a derivative(finite or infinite) at each interior point. Assume also that $f$ has finite one-sided derivatives $f'_{+}(a)$ and $f'_{-}(b)$ at the endpoints, with $f'_{+}(a)\neq f'_{-}(b)$. Then, if c is a real number between $f'_{+}(a)$ and $f'_{-}(b)$, there exists at least one interior point $x$ such that $f^\prime(x)=c$
\end{theorem}

\begin{proof}
proof here
\end{proof}





%\begin{align}
%$f'(x)$ &= \frac{f(x + h) - f(x)}{h} \\
%&= -\frac{\hbar^2}{2m}\nabla^2\psi(x,t) + V(x)\psi(x,t). \nonumber
%\end{align}
% Use align environments for equations. The \\ is a newline character. The & is the alignment character.
% Using align* or \nonumber on each line removes equation numbers

%\subsection{Subsection Title Here} % Specify subsections and subsubsections this way

%\begin{figure}[h!] % h forces the figure to be placed here, in the text
%  \includegraphics[width=0.4\textwidth]{stormtroopocat.jpg}  % if pdflatex is used, jpg, pdf, and png are permitted
%  \caption{The figure caption goes here.}
%  \label{fig:figlabel}
%\end{figure}

%This text should be below the figure unless \LaTeX  decides that a different layout works better.

% Repeat as needed
\end{document}
